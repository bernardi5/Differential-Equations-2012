%Time-stamp: <[2012course.tex] modified by Enrico Bernardi on Monday, 2012.03.12 at 15 : 32 : 47 on Enrico-Bernardis-MacBook-Pro.local>


\documentclass{beamer}
%test svn
%a test from laedell
%now a test again from ubuntu
%now again from lapdell
\mode<presentation>
{
 % \usetheme{Warsaw}
  % or ...

 % \setbeamercovered{transparent}
  % or whatever (possibly just delete it)
}
\usepackage{hyperref}
%
\usepackage{mathrsfs}
%\usepackage{showkeys}
\usepackage{amsmath}
\usepackage{amssymb}
\usepackage{yhmath}
\DeclareMathOperator{\Char}{Char}
\DeclareMathOperator{\spt}{Sp}
\DeclareMathOperator{\im}{Im}
\DeclareMathOperator{\sgn}{sgn}
\DeclareMathOperator{\re}{Re}
%\DeclareMathOperator{\dim}{dim}
\DeclareMathOperator{\rk}{rank}
%\DeclareMathOperator{\ker}{Ker}


\usepackage[english]{babel}
% or whatever

\usepackage[latin1]{inputenc}
% or whatever
\usepackage[LY1]{fontenc}               % specify text font encoding
%\usepackage[LY1,mtbold]{mathtime}       % switch math fonts
%\usepackage{times}
\usepackage{lucidabr}
%\usepackage[T1]{fontenc}
% Or whatever. Note that the encoding and the font should match. If T1
% does not look nice, try deleting the line with the fontenc.
\def\dif{\partial}
\def\hz{{\hat z}}
\def\hx{{\hat x}}
\def\hxi{{\hat \xi}}
\def\pf{\noindent Proof:\;}
\def\R{{\mathbb R}}
\def\C{{\mathbb C}}
\def\N{{\mathbb N}}
\def\K{{\mathbb K}}
\def\brho{{\bar\rho}}
\def\qed{\hfill{q.e.d.}\vspace{2mm}}
\def\lr#1{\langle{#1}\rangle}
\def\Re{{\rm Re}}
\def\x#1{x^{(#1)}}
\def\xx#1{\xi^{(#1)}}
\def\xip{\xi'}
\def\intro{Prerequisites from Linear Algebra and Matrix
    Theory}


% \title[QUANTITATIVE METHODS FOR ECONOMIC ANALYSIS] % (optional, use only with long paper titles)
% {QUANTITATIVE METHODS FOR ECONOMIC ANALYSIS}

% \subtitle
% {School of Economics, Bologna}

% \author[Enrico Bernardi] % (optional, use only with lots of authors)
% {E.~Bernardi\inst{1}}


% \institute[University of Bologna, School of Economics] % (optional, but mostly needed)
% {
%   \inst{1}%
%   Department Matemates\\
%   University of Bologna

% }
% - Keep it simple, no one is interested in your street address.

%\date[LMEC 2011] % (optional, should be abbreviation of conference name)
%{LMEC First Quarter, 2011}


%\subject{Mathematics}




% If you wish to uncover everything in a step-wise fashion, uncomment
% the following command: 

%\beamerdefaultoverlayspecification{<+->}


\begin{document}

% \begin{frame}
%   \titlepage
% \end{frame}

\begin{frame}
  \frametitle{Prerequisites from Linear Algebra and Matrix Theory}
$ \S 1$\textbf{Finite-Dimensional Complex Linear Spaces and Matrices}.

\textit{1.1 Complex Linear $ n $-space}.

We define an $ n$-vector $ x $ to be an ordered $ n $-tuple of complex
numbers $ \xi_{1},\xi_{2}, \dots \xi_{n} $ which we write as a column:
\begin{equation*}
  x = \begin{bmatrix}
 \xi_{1}  \\[2pt] 
 \ldots \\[2pt] 
 \xi_{n}  \\[2pt] 
\end{bmatrix}
\end{equation*}

The set of all such vectors is denoted by $ \C^{n} $.
We denote complex conjugation by a bar:($ \bar{z} = a -ib, z=a+ib) $

\begin{equation*}
  \bar{x} = \begin{bmatrix}
 \bar{\xi}_{1}  \\[2pt] 
 \ldots \\[2pt] 
 \bar{\xi}_{n}  \\[2pt] 
\end{bmatrix}
\end{equation*}
 
\end{frame}


\frame{ \frametitle{Prerequisites from Linear Algebra and Matrix Theory}%
%
The vector with zero components will be denoted by $ 0 $. The
transpose of  an $ n$-vector $ x $ is an ordered $ n $-tuple of complex
numbers written as a row:
\begin{equation*}
  x^{t} = (\xi_{1},\xi_{2}, \ldots, \xi_{n})
\end{equation*}

The Hermitian conjugate of $ x $ is defined by
\begin{equation*}
  x^{*} = \overline{x^{t}} = (\bar{\xi}_{1},\bar{\xi}_{2}, \ldots, \bar{\xi}_{n})
\end{equation*}

The vectors $ x + y $ and $ \lambda x $, where $ \lambda $ is a
complex number are defined by
\begin{equation*}
 \begin{bmatrix}
 \xi_{1} + \eta_{1}  \\[2pt] 
 \ldots \\[2pt] 
 \xi_{n} + \eta_{n}  \\[2pt] 
\end{bmatrix} \quad \mathrm{and} \quad  \begin{bmatrix}
\lambda \eta_{1}  \\[2pt] 
 \ldots \\[2pt] 
 \lambda\eta_{n}  \\[2pt] 
\end{bmatrix}
\end{equation*}

%
}


\frame{ \frametitle{Prerequisites from Linear Algebra and Matrix Theory}%
%
Addition of vectors and multiplication of vectors by complex numbers
satisfy the usual rules:

\begin{eqnarray*}
  x+0=x, x+y =y+x, x +(y+z) = (x+y) +z, 1\times x = x\\
\alpha(x+y) = \alpha x + \alpha y, (\alpha + \beta)x = \alpha x +
\beta x, \alpha(\beta x) = (\alpha \beta)x\\
\end{eqnarray*}
%

As usual $ (x,y) = y^{*}x $ is the scalar product

%
\begin{equation}
\label{eq1}
(x,y) = y^{*}x = \xi_{1}\bar{\eta}_{1} + \ldots \xi_{n}\bar{\eta}_{n}
\end{equation}
%
}


\frame{ \frametitle{Prerequisites from Linear Algebra and Matrix
    Theory}%
It follows from the definition (\ref{eq1}) that

\begin{eqnarray*}
  (y,x) = \overline{(x,y)}, (\alpha x, \beta y) =
  \alpha\bar{\beta}(x,y)\\
(x + y,z) = (x,z) + (y,z), (x, y + z) = (x,y) + (x,z)\\
(x,x) = |\xi_{1}|^{2} + \ldots +|\xi_{n}|^{2} > 0, x\neq 0\\
\end{eqnarray*}

The norm (modulus, length) of a vector $ x \in \C^{n}  $ is denoted by
$ |x| $:

\begin{equation}
\label{eq2}
  |x| = \sqrt{(x,x)} = \sqrt{|\xi_{1}|^{2} + \ldots +|\xi_{n}|^{2}}
\end{equation}
}


\frame{ \frametitle{\intro}%
%
The norm has the following properties:
\begin{enumerate}
\item $ |x| > 0 $ if $ x \neq 0 $
\item $ |\bar{x}| = |x| $
\item $ |\lambda x| = |\lambda| |x| $ where $ \lambda \in \C $ is a
  scalar
\item $ | x + y| \leq |x| + |y| $ (triangle inequality)
\item $ |(x,y)| \leq |x| |y| $ (Cauchy-Schwartz-Bunyakovskii inequality)
\end{enumerate}
%
}


\frame{ \frametitle{\intro}%
%
In general let $ V $ a vector space on the field $ \K $. ($ \K = \R $
or $ \K = \C $). We have the following:

\begin{definition}
  A norm on $ V $ is an application $ x \mapsto |x| $ from $ V $ to $
  [0,+\infty[ $ such that for any vectors $ x,y \in V $ and for every
  scalar $ \lambda \in \K $ we have:
\end{definition}
%

\begin{itemize}
  \item $ |x| = 0 \Rightarrow x=0$
\item $ |\lambda x| = |\lambda||x| $ homogeneity
\item $ |x + y| \leq |x| + |y| $ triangle inequality
\end{itemize}


We then say that $ V $ is a \textbf{normed} space on $ \K $.
}
**
\frame{ \frametitle{\intro}%
%
Let $ p\in\N $ be a natural number $ \geq 1 $. For $z\in\C^{n}$ the
following also define norms on $ \C^{n} $:
\begin{eqnarray*}
  |z|_{\infty} = \max_{j=1}^{n}|z_{j}|, \qquad |z|_{p} =\left(\sum_{j=1}^{n}|z_{j}|^{p}\right)^{\frac{1}{p}}
\end{eqnarray*}

$ |\bullet|_{\infty} $ is called the \textit{maximum norm}, $
|\bullet|_{p} $ is called the $ p $- norm. The norm in (\ref{eq2}) is
the $ 2 $- norm. All norms in $ \C^{n} $ define the same topology.This is a consequence of the fact that, as we will show now, in finite dimensional space all norms are equivalent.

%
}

\frame{ \frametitle{\intro}%
%
\begin{definition}
Two norms $ N_{1}, N_{2} $ on a vector space $ V $ are called
equivalent, if there are constants $ c,c'>0 $ such that
\end{definition}
\begin{eqnarray*}
  cN_{1}(x) \leq N_{2}(x) \leq c'N_{1}(x) \quad \mathrm{for} ~ \mathrm{all }
  \quad x \in V
\end{eqnarray*}
%
}


\frame{ \frametitle{\intro}%
%
\textbf{Proposition}
  
   On a finite-dimensional vector space $ V $ (over $\R$  or $\C$) all norms
are equivalent.
  
%
}
**


\end{document}


