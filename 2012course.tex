%Time-stamp: <[2012course.tex] modified by Enrico Bernardi on Saturday, 2012.03.10 at 18 : 09 : 56 on Enrico-Bernardis-MacBook-Pro.local>


\documentclass{beamer}
%test svn
%a test from laedell
%now a test again from ubuntu
%now again from lapdell
\mode<presentation>
{
 % \usetheme{Warsaw}
  % or ...

 % \setbeamercovered{transparent}
  % or whatever (possibly just delete it)
}
\usepackage{hyperref}
%
\usepackage{mathrsfs}
%\usepackage{showkeys}
\usepackage{amsmath}
\usepackage{amssymb}
\usepackage{yhmath}
\DeclareMathOperator{\Char}{Char}
\DeclareMathOperator{\spt}{Sp}
\DeclareMathOperator{\im}{Im}
\DeclareMathOperator{\sgn}{sgn}
\DeclareMathOperator{\re}{Re}
%\DeclareMathOperator{\dim}{dim}
\DeclareMathOperator{\rk}{rank}
%\DeclareMathOperator{\ker}{Ker}


\usepackage[english]{babel}
% or whatever

\usepackage[latin1]{inputenc}
% or whatever
\usepackage[LY1]{fontenc}               % specify text font encoding
%\usepackage[LY1,mtbold]{mathtime}       % switch math fonts
%\usepackage{times}
\usepackage{lucidabr}
%\usepackage[T1]{fontenc}
% Or whatever. Note that the encoding and the font should match. If T1
% does not look nice, try deleting the line with the fontenc.
\def\dif{\partial}
\def\hz{{\hat z}}
\def\hx{{\hat x}}
\def\hxi{{\hat \xi}}
\def\pf{\noindent Proof:\;}
\def\R{{\mathbb R}}
\def\C{{\mathbb C}}
\def\N{{\mathbb N}}
\def\brho{{\bar\rho}}
\def\qed{\hfill{q.e.d.}\vspace{2mm}}
\def\lr#1{\langle{#1}\rangle}
\def\Re{{\rm Re}}
\def\x#1{x^{(#1)}}
\def\xx#1{\xi^{(#1)}}
\def\xip{\xi'}



% \title[QUANTITATIVE METHODS FOR ECONOMIC ANALYSIS] % (optional, use only with long paper titles)
% {QUANTITATIVE METHODS FOR ECONOMIC ANALYSIS}

% \subtitle
% {School of Economics, Bologna}

% \author[Enrico Bernardi] % (optional, use only with lots of authors)
% {E.~Bernardi\inst{1}}


% \institute[University of Bologna, School of Economics] % (optional, but mostly needed)
% {
%   \inst{1}%
%   Department Matemates\\
%   University of Bologna

% }
% - Keep it simple, no one is interested in your street address.

%\date[LMEC 2011] % (optional, should be abbreviation of conference name)
%{LMEC First Quarter, 2011}


%\subject{Mathematics}




% If you wish to uncover everything in a step-wise fashion, uncomment
% the following command: 

%\beamerdefaultoverlayspecification{<+->}


\begin{document}

% \begin{frame}
%   \titlepage
% \end{frame}

\begin{frame}
  \frametitle{Prerequisites from Linear Algebra and Matrix Theory}
Finite-Dimensional Complex Linear Spaces and Matrices.

Complex Linear $ n $-space.

We define an $ n$-vector $ x $ to be an ordered $ n $-tuple of complex
numbers $ \xi_{1},\xi_{2}, \dots \xi_{n} $ which we write as a column:
\begin{equation*}
  x = \begin{bmatrix}
 \xi_{1}  \\[2pt] 
 \ldots \\[2pt] 
 \xi_{n}  \\[2pt] 
\end{bmatrix}
\end{equation*}

The set of all such vectors is denoted by $ \C^{n} $.
We denote complex conjugation by a bar:($ \bar{z} = a -ib, z=a+ib) $

\begin{equation*}
  \bar{x} = \begin{bmatrix}
 \bar{\xi}_{1}  \\[2pt] 
 \ldots \\[2pt] 
 \bar{\xi}_{n}  \\[2pt] 
\end{bmatrix}
\end{equation*}
 
\end{frame}


\frame{ \frametitle{Prerequisites from Linear Algebra and Matrix Theory}%
%
The vector with zero components will be denoted by $ 0 $. The
transpose of  an $ n$-vector $ x $ is an ordered $ n $-tuple of complex
numbers written as a row:
\begin{equation*}
  x^{t} = (\xi_{1},\xi_{2}, \ldots, x_{n})
\end{equation*}

The Hermitian conjugate of $ x $ is defined by
\begin{equation*}
  x^{*} = \overline{x^{t}} = (\bar{\xi}_{1},\bar{\xi}_{2}, \ldots, \bar{\xi}_{n})
\end{equation*}

The vectors $ x + y $ and $ \lambda x $, where $ \lambda $ is a
complex number are defined by
\begin{equation*}
 \begin{bmatrix}
 \xi_{1} + \eta_{1}  \\[2pt] 
 \ldots \\[2pt] 
 \xi_{n} + \eta_{n}  \\[2pt] 
\end{bmatrix} \quad \mathrm{and} \quad  \begin{bmatrix}
\lambda \eta_{1}  \\[2pt] 
 \ldots \\[2pt] 
 \lambda\eta_{n}  \\[2pt] 
\end{bmatrix}
\end{equation*}

%
}



\end{document}


